\documentclass{beamer}
\usepackage[brazilian]{babel}
\usepackage[utf8]{inputenc}
\usepackage{calc}
\usepackage[absolute,overlay]{textpos}
\mode<presentation>{\usetheme{tud}}

\title[JEWEL]{JEWEL}
%\subtitle
\institute[HEPIC-Instituto de Física]{HEPIC-Instituto de Física}
\author{Fabio Canedo}
\date{\today}

% Insert frame before each subsection (requires 2 latex runs)
\AtBeginSubsection[] {
	\begin{frame}<beamer>\frametitle{\titleSubsec}
		\tableofcontents[currentsection,currentsubsection]  % Generation of the Table of Contents
	\end{frame}
}
% Define the title of each inserted pre-subsection frame
\newcommand*\titleSubsec{Content}
% Define the title of the "Table of Contents" frame
\newcommand*\titleTOC{Content}

% define a symbol which can be removed if you don't need it
\newcommand{\field}[1]{\mathbb{#1}}
\newcommand{\Zset}{\field{Z}}

\begin{document}

{
% remove the next line if you don't want a background image
\usebackgroundtemplate{}%
\setbeamertemplate{footline}{\usebeamertemplate*{minimal footline}}
\frame{\titlepage}
}

{\setbeamertemplate{footline}{\usebeamertemplate*{minimal footline}}
\begin{frame}\frametitle{\titleTOC}
	\tableofcontents
\end{frame}
}

\section{Quick Review}
\subsection{What is JEWEL?}

\begin{frame}\frametitle{What is JEWEL?}
        JEWEL is a Monte Carlo High Energy generator whose name stands for
        Jet Evolution with Energy Loss. It propagates partons inside a
        thermal medium that simulates heavy-ion collisions in accelerators
        such as RHIC or LHC.
        \\
        \pause
        It accomplishes this by use of a perturbative approach, combining regular parton shower
        with successive scatterings:
        \begin{minipage}{1.0\textwidth}
		\includegraphics[scale=0.5]{images/feynman.png}        
        \end{minipage}
\end{frame}

\subsection{Event Generation}
\begin{frame}\frametitle{Event Generations}
	The process of event generation if described by the following:
	\begin{itemize}
	\pause
	\item First, PYTHIA is called to calculate the initial scatterings of the partons inside the
	nuclei;
	\pause
	\item The scattered partons are then propagated through the medium with JEWEL, where the medium
	is seen by the parton as a collection of scattering centres;
	\pause
	\item Once all of the partons have reached the minimum of virtuality, given by an infrared
	regulator, they are given back to PYTHIA for hadronization;
	\end{itemize}
\end{frame}

\subsection{Physics of JEWEL}

\begin{frame}\frametitle{Physics of JEWEL}
	The use of perturbative approach is dealt with by making use of a probabilistic interpretation
	of the Sudakov form factor:
	\begin{equation}
	S_a(t_h,t_c) = \exp \left\{ -\int_{t_c}^{t_h} \frac{dt}{t} \int_{z_{min}}^{z_{max}} dz \sum_b \frac{\alpha(k_{\bot}^2)}{2\pi} \hat{P}_{ab}(z) \right\}
	\end{equation}
	\pause
	Wich means we can take it as the probability that a parton $a$ emites no resolvable radiation
	between the scales $t_c$ and $t_h$.
\end{frame}

\begin{frame}\frametitle{Physics of JEWEL}
	The Sudakov form factor is used to perform the evolution by enforcing radiation until the
	parton reaches the $t_c$ \textit{cut-off}.
	\\ \pause
	Besides that, the medium, seen as a collection of scattering centers by the parton, also has
	the probability of interaction.
	\\ \pause
	This probability is given by the regular cross-section of a $2 \rightarrow 2$ process,given by:
	\begin{equation}
	\sigma_i(E,T)=\int_{0}^{|\hat{t}|_{max}(E,T)}d|\hat{t}|\int_{x_{min}(|\hat{t}|)}^{x_{max}(|\hat{t}|)} dx \sum_{j \in \{ q,\overline{q},g\}} f_{j}^{i}(x,\hat{t})\frac{\mathrm{d}\sigma}{\mathrm{d}\hat{t}}(x\hat{s},|\hat{t}|)
	\end{equation}
\end{frame}

\section{Medium Model}
\subsection{Initial Conditions}

\begin{frame}\frametitle{Initial Conditions}
	The model employed by JEWEL is a variant of Bjorken model that
	approachs the initial contitions by the definition of the nuclear thickness 
	function:
	\begin{equation}
	T(x,y) = \int dz \rho(x,y,z)
	\end{equation}
\end{frame}

\begin{frame}\frametitle{Initial Conditions}
	The nuclear thickness functions are then used to define a reduced thickness by:
	\begin{equation}
	\begin{split}
	n(b,x,y) &= T_A(x-\frac{b}{2},y) \left( 1-\exp\left( \sigma_{NN} T_B(x+\frac{b}{2},y) \right) \right) \\
	&+ T_B(x+\frac{b}{2},y) \left( 1-\exp\left( \sigma_{NN} T_A(x-\frac{b}{2},y) \right) \right)
	\end{split}
	\end{equation}
	Where $\sigma_{NN}$ is the nuclear cross-section and $b$ is the impact parameter.
\end{frame}

\begin{frame}\frametitle{Initial Conditions}
	This reduced thickness is then applied to take a map of the initial energy density:
	\begin{equation}
	\epsilon(x,y,b,\tau_i) = \epsilon_i \frac{n_{part}(x,y,b)}{<n_{part}>(b=0)}
	\end{equation}
	Where $<n_{part}>(b=0) \approx \frac{2 A}{\pi R_A}$.
\end{frame}

\begin{frame}\frametitle{Initial Conditions}
	The value of $\epsilon_i$ is determined by an initial temperature given as a parameter to
	JEWEL. This translation is made through the use of the relation $\epsilon_i \propto T_i^4$.
\end{frame}

\subsection{Medium Evolution}

\begin{frame}\frametitle{Medium Evolution}
	The medium evolution is performed analytically by use of:
	\begin{equation}
	\epsilon(x,y,b,\tau) = \epsilon_i (x,y,b,\tau_i) \left( \frac{\tau}{\tau_i} \right)^{-\frac{4}{3}}
	\end{equation}
	\pause
	This that the temperature evolves according to:
	\begin{equation}
	T(x,y,b,\tau) \propto \epsilon(x,y,b,\tau_i)^{1/4} \left( \frac{\tau}{\tau_i} \right)^{-\frac{4}{3}}
	\end{equation}
\end{frame}

\section{Jet Observables}
\subsection{JetAlgorithms}

\begin{frame}\frametitle{Jet Algorithms}
	Today, the main type of algorithm to cluster particles from a given event into jets,
	thus defining a jet, are the sequential recombination algorithms. They define a distance
	functions between particles:
	\pause
	\begin{equation}
	d_{ij} = \min(p_{ti}^{2p},p_{tj}^{2p}) \frac{\Delta R_{ij}^2}{R^2}
	\end{equation}
	\pause
	Then a process of iteration is realized, where the particles with minimum $d_{ij}$ are
	combined to form the given jets.
\end{frame}

\begin{frame}\frametitle{Jet Algorithms}
	Once the jets are found, one class of the possible observables are the jet shape observables.
	\pause
	\\
	These reflect mush of the evolution that occurs in the jet formation, as well as
	the process of hadronization.
\end{frame}

\subsection{SoftDrop Procedure}

\begin{frame}\frametitle{SoftDrop Procedure}
	One form of extracting information about jet shape is through the SoftDrop Procedure.
	\pause
	\\
	This is done by undoing the last step of the jet recombination.
	\pause
	\\
	The SoftDrop condition is:
	\begin{equation}
	\frac{\min(p_{T1},p_{T2})}{p_{T1}+p_{T2}} > z_{cut} \left(\frac{\Delta R_{12}}{R_0}\right)^{\beta}
	\end{equation}
\end{frame}

\begin{frame}\frametitle{SoftDrop Procedure}
	Once the condition has been applied, the quantity:
	\begin{equation}
	\frac{\min(p_{T1},p_{T2})}{p_{T1}+p_{T2}}
	\end{equation}
	Reflects an observable that is usually related to the first splitting of the parent parton.
	\pause
	\\
	This can be seen in the results of the fact that they follow the Altarelli-Parisi splitting
	functions.
\end{frame}

\section{JEWEL Results}
\subsection{JEWEL Validation}

\begin{frame}\frametitle{JEWEL Validation}
    \begin{minipage}{1\textwidth}
	The validation of JEWEL has been performed by comparing with exepriments such as:    
    \end{minipage}
    \begin{columns}
    \begin{column}{0.5\textwidth}
	\begin{minipage}[l]{0.5\textwidth}
	\includegraphics[scale=0.5]{images/atlas_jewel.png}
	\end{minipage}
	\end{column}
    \begin{column}{0.5\textwidth}
	\begin{minipage}[r]{1\textwidth}
	Centrality dependence of the angular distribution of single inclusive jets in Pb+Pb collisions
	at $\sqrt{s_{NN}}=2.76\mathrm{TeV}$ for a jet radius $R=0.2$ and $|\eta_{jet}|<2.1$ in the
	range $45\mathrm{GeV}<p_t<60\mathrm{GeV}$.
	\end{minipage}
	\end{column}
	\end{columns}
\end{frame}

\begin{frame}\frametitle{JEWEL Validation}
    \begin{minipage}{1\textwidth}
	The validation of JEWEL has been performed by comparing with exepriments such as:    
    \end{minipage}
    \begin{columns}
    \begin{column}{0.5\textwidth}
	\begin{minipage}[l]{0.5\textwidth}
	\includegraphics[scale=0.5]{images/atlas_jewel.png}
	\end{minipage}
	\end{column}
    \begin{column}{0.5\textwidth}
	\begin{minipage}[r]{1\textwidth}
	Centrality dependence of the angular distribution of single inclusive jets in Pb+Pb collisions
	at $\sqrt{s_{NN}}=2.76\mathrm{TeV}$ for a jet radius $R=0.2$ and $|\eta_{jet}|<2.1$ in the
	range $45\mathrm{GeV}<p_t<60\mathrm{GeV}$.
	\end{minipage}
	\end{column}
	\end{columns}
\end{frame}

\begin{frame}\frametitle{JEWEL Validation}
    \begin{minipage}{1\textwidth}
	The validation of JEWEL has been performed by comparing with exepriments such as:    
    \end{minipage}
    \begin{columns}
    \begin{column}{0.5\textwidth}
	\begin{minipage}[l]{0.5\textwidth}
	\includegraphics[scale=0.5]{images/atlas_jewel_1.png}
	\end{minipage}
	\end{column}
    \begin{column}{0.5\textwidth}
	\begin{minipage}[r]{1\textwidth}
	Centrality dependence of the angular distribution of single inclusive jets in Pb+Pb collisions
	at $\sqrt{s_{NN}}=2.76\mathrm{TeV}$ for a jet radius $R=0.2$ and $|\eta_{jet}|<2.1$ in the
	range $45\mathrm{GeV}<p_t<60\mathrm{GeV}$.
	\end{minipage}
	\end{column}
	\end{columns}
\end{frame}

\begin{frame}\frametitle{JEWEL Validation}
    \begin{minipage}{1\textwidth}
	The validation of JEWEL has been performed by comparing with exepriments such as:    
    \end{minipage}
    \begin{columns}
    \begin{column}{0.5\textwidth}
	\begin{minipage}[l]{0.5\textwidth}
	\includegraphics[scale=0.2]{images/alice_jewel.png}
	\end{minipage}
	\end{column}
    \begin{column}{0.5\textwidth}
	\begin{minipage}[r]{1\textwidth}
	Jet shape distributions in $0\backsim10$ central PbPb collisions at $\sqrt{s_{NN}}=2.76\mathrm{TeV}$ for $R=0.2$ in range of jet $p_{chT}$,jet of $40\backsim60\mathrm{GeV}/c$ compared to JEWEL with and without recoils with different subtraction methods. The coloured boxes represent the experimental uncertainty on the jet shapes.
	\end{minipage}
	\end{column}
	\end{columns}
\end{frame}

\section{What to do with JEWEL?}
\subsection{What to do with JEWEL?}

\begin{frame}\frametitle{What to do with JEWEL?}
	The steps suggested of what to do next with JEWEL are:
	\begin{itemize}
	\pause
	\item Prepare a code to read external medium files(currently running);
	\pause
	\item Couple it with a different model of initial conditions(currently running with TRENTO);
	\pause
	\item Couple it with a more realistic hydro;
	\pause
	\item Talk to JEWEL developers to adapt it to run with haevy partons;
	\end{itemize}
\end{frame}

\end{document}
