\documentclass{beamer}
\usepackage[brazilian]{babel}
\usepackage[utf8]{inputenc}
\usepackage{calc}
\usepackage[absolute,overlay]{textpos}
%\mode<presentation>{\usetheme{tud}}

%\title[JEWEL]{JEWEL}
%\subtitle
%\institute[HEPIC-Instituto de Física]{HEPIC-Instituto de Física}
%\author{Fabio Canedo}
%\date{\today}

% Insert frame before each subsection (requires 2 latex runs)
%\AtBeginSubsection[] {
%	\begin{frame}<beamer>\frametitle{\titleSubsec}
%		\tableofcontents[currentsection,currentsubsection]  % Generation of the Table of Contents
%	\end{frame}
%}
% Define the title of each inserted pre-subsection frame
\newcommand*\titleSubsec{Content}
% Define the title of the "Table of Contents" frame
\newcommand*\titleTOC{Content}

% define a symbol which can be removed if you don't need it
\newcommand{\field}[1]{\mathbb{#1}}
\newcommand{\Zset}{\field{Z}}

\begin{document}

{
% remove the next line if you don't want a background image
%\usebackgroundtemplate{}%
%\setbeamertemplate{footline}{\usebeamertemplate*{minimal footline}}
%\frame{\titlepage}
}

{%\setbeamertemplate{footline}{\usebeamertemplate*{minimal footline}}
\begin{frame}\frametitle{}
	\begin{equation*}
      \nabla \cdot \overrightarrow{E} = \frac{\rho}{\epsilon_0}
    \end{equation*}
    \begin{equation*}
      \nabla \times \overrightarrow{B} = - \frac{\partial \overrightarrow{E}}{\partial t}
	\end{equation*}
	\begin{equation*}
	  \nabla \cdot \overrightarrow{B} = 0
	\end{equation*}
	\begin{equation*}
	  \nabla \times \overrightarrow{E} = \mu_0 \overrightarrow{j} + \frac{1}{c^{2}} \frac{\partial \overrightarrow{B}}{\partial t}
	\end{equation*}
\end{frame}
}




\end{document}
